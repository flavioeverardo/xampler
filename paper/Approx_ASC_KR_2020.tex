\def\year{2020}\relax
%File: formatting-instruction.tex
\documentclass[letterpaper]{article} % DO NOT CHANGE THIS
\usepackage{aaai20}  % DO NOT CHANGE THIS
\usepackage{times}  % DO NOT CHANGE THIS
\usepackage{helvet} % DO NOT CHANGE THIS
\usepackage{courier}  % DO NOT CHANGE THIS
\usepackage[hyphens]{url}  % DO NOT CHANGE THIS
\usepackage{graphicx} % DO NOT CHANGE THIS
\urlstyle{rm} % DO NOT CHANGE THIS
\def\UrlFont{\rm}  % DO NOT CHANGE THIS
\usepackage{graphicx}  % DO NOT CHANGE THIS
\frenchspacing  % DO NOT CHANGE THIS
\setlength{\pdfpagewidth}{8.5in}  % DO NOT CHANGE THIS
\setlength{\pdfpageheight}{11in}  % DO NOT CHANGE THIS
%\nocopyright

% added by Ankit
\usepackage{xspace}
\def\approxmc{\textsf{ApproxMC}\xspace}
\def\approxasc{\textsf{ApproxASC}\xspace}
\def\asp{\textsf{\#ASP}\xspace}

% Copied from Flavio draft
\usepackage{url}
\usepackage{microtype}
\usepackage{hyperref}
\usepackage{float}
\usepackage{listings}

\usepackage[ruled,vlined]{algorithm2e}
\SetKwComment{Comment}{$\triangleright$\ }{}

\newcommand{\XOR}{\textsc{xor}} % just an idea


% - systems ----------------------------------------------------------------------
% >>> NOT USED in BOOK <<<
\newcommand{\sysfont}{\textit}
\newcommand{\acthex}{\sysfont{acthex}}
\newcommand{\adsolver}{\sysfont{adsolver}}
\newcommand{\anthem}{\sysfont{anthem}}
\newcommand{\asparagus}{\sysfont{asparagus}}
\newcommand{\aspartame}{\sysfont{aspartame}}
\newcommand{\aspic}{\sysfont{aspic}}
\newcommand{\aspcud}{\sysfont{aspcud}}
\newcommand{\aspmt}{\sysfont{aspmt}}
\newcommand{\asprilo}{\sysfont{asprilo}}
\newcommand{\asprin}{\sysfont{asprin}}
\newcommand{\assat}{\sysfont{assat}}
\newcommand{\berkmin}{\sysfont{berkmin}}
\newcommand{\chuffed}{\sysfont{chuffed}}
\newcommand{\chasp}{\sysfont{chasp}}
\newcommand{\claspD}{\sysfont{claspD}}
\newcommand{\claspar}{\sysfont{claspar}}
\newcommand{\claspfolio}{\sysfont{claspfolio}}
\newcommand{\clasp}{\sysfont{clasp}}
\newcommand{\claspre}{\sysfont{claspre}}
\newcommand{\clingcon}{\sysfont{clingcon}}
\newcommand{\clingo}{\sysfont{clingo}}
\newcommand{\clingod}[1]{\clingo\textnormal{[}\textsc{#1}\textnormal{]}}
\newcommand{\cmodels}{\sysfont{cmodels}}
\newcommand{\coala}{\sysfont{coala}}
\newcommand{\cplex}{\sysfont{cplex}}
\newcommand{\dingo}{\sysfont{dingo}}
\newcommand{\dflat}{\sysfont{dflat}}
\newcommand{\dlvhex}{\sysfont{dlvhex}}
\newcommand{\dlv}{\sysfont{dlv}}
\newcommand{\ezcsp}{\sysfont{ezcsp}}
\newcommand{\ezsmt}{\sysfont{ezsmt}}
\newcommand{\embasp}{\sysfont{embasp}}
\newcommand{\fastdownward}{\sysfont{fastdownward}}
\newcommand{\ftolp}{\sysfont{f2lp}}
\newcommand{\gasp}{\sysfont{gasp}}
\newcommand{\gfd}{\sysfont{g12fd}}
\newcommand{\gecode}{\sysfont{gecode}}
\newcommand{\gidl}{\sysfont{gidl}}
\newcommand{\gnt}{\sysfont{gnt}}
\newcommand{\gringo}{\sysfont{gringo}}
\newcommand{\harvey}{\sysfont{harvey}}
\newcommand{\iclingo}{\sysfont{iclingo}}
\newcommand{\idp}{\sysfont{idp}}
\newcommand{\inca}{\sysfont{inca}}
\newcommand{\jdlv}{\sysfont{jdlv}}
\newcommand{\lctocasp}{\sysfont{lc2casp}}
\newcommand{\lparse}{\sysfont{lparse}}
\newcommand{\lpsolve}{\sysfont{lpsolve}}
\newcommand{\lptodiff}{\sysfont{lp2diff}}
\newcommand{\lptosat}{\sysfont{lp2sat}}
\newcommand{\mchaff}{\sysfont{mchaff}}
\newcommand{\measp}{\sysfont{measp}}
\newcommand{\metasp}{\sysfont{metasp}}
\newcommand{\mingo}{\sysfont{mingo}}
\newcommand{\minisat}{\sysfont{minisat}}
\newcommand{\minisatid}{\sysfont{minisatid}}
\newcommand{\minizinc}{\sysfont{minizinc}}
\newcommand{\mznfzn}{\sysfont{mzn2fzn}}
\newcommand{\nomorepp}{\sysfont{nomore++}}
\newcommand{\oclingo}{\sysfont{oclingo}}
\newcommand{\omiga}{\sysfont{omiga}}
\newcommand{\piclasp}{\sysfont{piclasp}}
\newcommand{\picosat}{\sysfont{picosat}}
\newcommand{\picat}{\sysfont{picat}}
\newcommand{\picatsat}{\sysfont{picatsat}}
\newcommand{\plasp}{\sysfont{plasp}}
\newcommand{\quontroller}{\sysfont{quontroller}}
\newcommand{\reify}{\sysfont{reify}}
\newcommand{\rosoclingo}{\sysfont{rosoclingo}}
\newcommand{\sag}{\sysfont{sag}}
\newcommand{\satz}{\sysfont{satz}}
\newcommand{\siege}{\sysfont{siege}}
\newcommand{\smac}{\sysfont{smac}}
\newcommand{\smodelscc}{\sysfont{smodels$_{\!cc}$}}
\newcommand{\smodelsr}{\sysfont{smodels}$_r$}
\newcommand{\smodels}{\sysfont{smodels}}
\newcommand{\sugar}{\sysfont{sugar}}
\newcommand{\teaspoon}{\sysfont{teaspoon}}
\newcommand{\tel}{\sysfont{tel}}
\newcommand{\telingo}{\sysfont{telingo}}
\newcommand{\unclasp}{\sysfont{unclasp}}
\newcommand{\wasp}{\sysfont{wasp}}
\newcommand{\xorro}{\sysfont{xorro}}
\newcommand{\zchaff}{\sysfont{zchaff}}
\newcommand{\zzz}{\sysfont{z3}}
\newcommand{\cryptominisat}{\sysfont{crypto-minisat}}

\newcommand{\clingoM}[1]{\clingo{\small\textnormal{[}\textsc{#1}\textnormal{]}}}
\newcommand{\ASPm}[1]{ASP\raisebox{.7pt}{[\textsc{#1}]}}

\newcommand{\flatzinc}{\sysfont{FlatZinc}}
\newcommand{\aspif}{\sysfont{aspif}}

\newcommand{\python}{Python}
\newcommand{\lua}{Lua}
\newcommand{\cpp}{C++}
\newcommand{\C}{C}
\newcommand{\java}{Java}
\newcommand{\haskell}{Haskell}

\newcommand{\el}[1]{\vspace{#1\baselineskip}}

% Some notational macros

\newcommand{\rg}[3]{{#1}{#2}\,\ldots{#2}\,{#3}}
\newcommand{\xor}{\oplus}
\newcommand{\true}{\top}
\newcommand{\false}{\bot}
\newcommand{\naf}{\mathit{not}\,}
\newcommand{\set}[1]{\{#1\}}

% Tomi's specialties (not needed in the end)

\newcommand{\TODO}[1]{%
\vspace{1\baselineskip}\noindent\hrulefill\hspace{1em}{#1}%
\hspace{1em}\hrulefill\vspace{1\baselineskip}}

\newcommand{\DONE}{%
\vspace{1\baselineskip}\noindent\hrule\vspace{1\baselineskip}}

%%% Local Variables:
%%% mode: latex
%%% TeX-master: "paper"
%%% End:



%PDF Info Is REQUIRED.
% For /Author, add all authors within the parentheses, separated by commas. No accents or commands.
% For /Title, add Title in Mixed Case. No accents or commands. Retain the parentheses.
 \pdfinfo{
/Title (A Scalable Approximate Model Counter for ASP)
/Author (AAAI Press Staff, Pater Patel Schneider, Sunil Issar, J. Scott Penberthy, George Ferguson, Hans Guesgen)
} %Leave this	
% /Title ()
% Put your actual complete title (no codes, scripts, shortcuts, or LaTeX commands) within the parentheses in mixed case
% Leave the space between \Title and the beginning parenthesis alone
% /Author ()
% Put your actual complete list of authors (no codes, scripts, shortcuts, or LaTeX commands) within the parentheses in mixed case. 
% Each author should be only by a comma. If the name contains accents, remove them. If there are any LaTeX commands, 
% remove them. 

% DISALLOWED PACKAGES
% \usepackage{authblk} -- This package is specifically forbidden
% \usepackage{balance} -- This package is specifically forbidden
% \usepackage{caption} -- This package is specifically forbidden
% \usepackage{color (if used in text)
% \usepackage{CJK} -- This package is specifically forbidden
% \usepackage{float} -- This package is specifically forbidden
% \usepackage{flushend} -- This package is specifically forbidden
% \usepackage{fontenc} -- This package is specifically forbidden
% \usepackage{fullpage} -- This package is specifically forbidden
% \usepackage{geometry} -- This package is specifically forbidden
% \usepackage{grffile} -- This package is specifically forbidden
% \usepackage{hyperref} -- This package is specifically forbidden
% \usepackage{navigator} -- This package is specifically forbidden
% (or any other package that embeds links such as navigator or hyperref)
% \indentfirst} -- This package is specifically forbidden
% \layout} -- This package is specifically forbidden
% \multicol} -- This package is specifically forbidden
% \nameref} -- This package is specifically forbidden
% \natbib} -- This package is specifically forbidden -- use the following workaround:
% \usepackage{savetrees} -- This package is specifically forbidden
% \usepackage{setspace} -- This package is specifically forbidden
% \usepackage{stfloats} -- This package is specifically forbidden
% \usepackage{tabu} -- This package is specifically forbidden
% \usepackage{titlesec} -- This package is specifically forbidden
% \usepackage{tocbibind} -- This package is specifically forbidden
% \usepackage{ulem} -- This package is specifically forbidden
% \usepackage{wrapfig} -- This package is specifically forbidden
% DISALLOWED COMMANDS
% \nocopyright -- Your paper will not be published if you use this command
% \addtolength -- This command may not be used
% \balance -- This command may not be used
% \baselinestretch -- Your paper will not be published if you use this command
% \clearpage -- No page breaks of any kind may be used for the final version of your paper
% \columnsep -- This command may not be used
% \newpage -- No page breaks of any kind may be used for the final version of your paper
% \pagebreak -- No page breaks of any kind may be used for the final version of your paperr
% \pagestyle -- This command may not be used
% \tiny -- This is not an acceptable font size.
% \vspace{- -- No negative value may be used in proximity of a caption, figure, table, section, subsection, subsubsection, or reference
% \vskip{- -- No negative value may be used to alter spacing above or below a caption, figure, table, section, subsection, subsubsection, or reference

\setcounter{secnumdepth}{0} %May be changed to 1 or 2 if section numbers are desired.

% The file aaai20.sty is the style file for AAAI Press 
% proceedings, working notes, and technical reports.
%
\setlength\titlebox{2.5in} % If your paper contains an overfull \vbox too high warning at the beginning of the document, use this
% command to correct it. You may not alter the value below 2.5 in
\title{A Scalable Approximate Answer Sets Counter}
%Your title must be in mixed case, not sentence case. 
% That means all verbs (including short verbs like be, is, using,and go), 
% nouns, adverbs, adjectives should be capitalized, including both words in hyphenated terms, while
% articles, conjunctions, and prepositions are lower case unless they
% directly follow a colon or long dash
\author{Flavio Everardo\textsuperscript{\rm 1}, Johannes Fichte\textsuperscript{\rm 2}, Markus Hecher\textsuperscript{\rm 3}, Kuldeep Meel\textsuperscript{\rm 4}, Ankit Shukla\textsuperscript{\rm 5} \\ % All authors must be in the same font size and format. Use \Large and \textbf to achieve this result when breaking a line
\textsuperscript{\rm 1} University of Potsdam, Germany\\ 
\textsuperscript{\rm 2} TU Dresden, Germany\\ 
\textsuperscript{\rm 3}  TU Wien, Vienna, Austria \\ \textsuperscript{\rm 4} National University of Singapore, Singapore \\ \textsuperscript{\rm 5} JKU, Linz, Austria
%If you have multiple authors and multiple affiliations
% use superscripts in text and roman font to identify them. For example, Sunil Issar,\textsuperscript{\rm 2} J. Scott Penberthy\textsuperscript{\rm 3} George Ferguson,\textsuperscript{\rm 4} Hans Guesgen\textsuperscript{\rm 5}. Note that the comma should be placed BEFORE the superscript for optimum readability % email address must be in roman text type, not monospace or sans serif
}
 \begin{document}

\maketitle

\begin{abstract}
(\asp) XXX
\end{abstract}

\section{TODOS} \label{sec:todos}
\begin{enumerate}

\item Ankit: 
  \begin{enumerate}
  \item Adapt the basic \approxmc algorithm to \asp.
  \item Collect all the relevant literature on the Counting Answer sets. Approx or Exact. 
  \end{enumerate}
\end{enumerate}


\section{Introduction} \label{sec:introduction}
 Parity constraints constitute the foundations of reasoning modes like sampling or (approximate) model counting \cite{DBLP:journals/corr/abs-1806-02239}, as well as circuit verification and cryptography \cite{DBLP:conf/laitinen2014extending}.
 %
 With most of their applications in the neighboring area of Satisfiability Testing (SAT) \cite{DBLP:journals/corr/abs-1806-02239},
 almost no attention has so far been paid to their integration into Answer Set Programming (ASP).%~\cite{kalepesc16a}.
 
 Previous efforts represented parity constraints into ASP in three ways,
 via the \texttt{\#count} aggregate coupled with a modulo-two operation as used or sampling in the initial prototype of \xorro{} from 2009\footnote{https://sourceforge.net/p/potassco/code/HEAD/tree/branches/xorro},
 as lists, as shown in \emph{harvey}~\cite{DBLP:conf/lpnmr/GresslerOT17},
 and as the (discontinued) aggregates \texttt{\#even} and \texttt{\#odd} from \gringo{} series 3 via meta-encodings.
 %
 Unlike these approaches, several SAT solvers feature rather sophisticated treatments of parity constraints.
 For instance, most popularly the award-winning solver \cryptominisat~\cite{DBLP:conf/sat/SoosNC09}, which pursues a hybrid approach,
 addressing parity constraints separately with Gauss-Jordan Elimination (GJE).
 
 To this end, we present the next generation of \xorro,~\cite{DBLP:conf/lpnmr/EverardoJKS19}
 implementing six alternatives to handle parity constraints into ASP,
 benefiting from the advanced interfaces of \clingo{},
 and the sophisticated solving techniques developed in SAT.
 %
 We propose two types of approaches, eager and lazy.~\footnote{Both eager and lazy follows the methodology from Satisfiability modulo theories.}
 The former relies on ASP encodings of parity constraints, and the latter uses theory propagators within \clingo{}’s Python interface~\cite{DBLP:conf/iclp/GebserKKOSW16}.
 
 To accommodate parity constraints in the input language,
 we rely on \clingo’s theory language extension~\cite{DBLP:conf/iclp/GebserKKOSW16}
 following the common syntax of \emph{aggregates}~\cite{DBLP:journals/tplp/GebserHKLS15}:
 
 %
 % ------------------------------------------------------------------------------
 \lstinputlisting[basicstyle=\small\ttfamily]{listings/xor_constraints_ex1.lp}
 % ------------------------------------------------------------------------------
 %
 That is, \xorro{} extends the input language of \clingo{} by
 aggregate names \texttt{\&even} and \texttt{\&odd} that are followed
 by a set, whose elements are \emph{terms} conditioned by
 conjunctions of literals separated by commas.%
 %
 \footnote{In turn, multiple conditional terms within an aggregate are
 	separated by semicolons.}
 %
 In the context of a choice rule \texttt{\set{p(1..3)}.}, the parity constraints shown above
 amounts to the \XOR{} operations: \\
 $p(1) \xor \bot$ and $p(2) \xor p(3) \xor \top$
 (where $\bot$ and $\top$ stand for the Boolean constants true and false, respectively)
 yielding the answer sets~\texttt{\set{p(1)}} and \texttt{\set{p(1),p(2),p(3)}}.
 
 
 %The semantics of aggregates formed with keywords \texttt{\&even} and
 %\texttt{\&odd} is defined by \emph{even} and \emph{odd} parity constraints,
 %respectively.
 %In this implementation,
 Currently, these constraints are interpreted as directives,
 filtering answer sets that do not satisfy the parity constraint in question.
 \footnote{For now, parity constraints may not occur in the bodies nor the heads of rules.}
 %
 Hence, the first
 constraint filters out answer sets not containing the atom
 \texttt{p(1)}, while the second requires that either none or both of
 the atoms \texttt{p(2)} and \texttt{p(3)} are included.
 
 %As mentioned, \xorro{} handles parity constraints in six ways as shown in Table \ref{table:xorro_approaches},
 %implements different means to handle parity constraints as shown in Table \ref{table:xorro_approaches},
 %where the first three corresponds to the eager, and the last three to the lazy approaches.
 Table~ \ref{table:xorro_approaches}, shows the six implementations to handle parity constraints.
 The first three corresponds to the eager, and the last three to the lazy approaches. 
 
 \begin{table*}[t]
% 	\centering
\caption{\xorro{} approaches to handle parity constraints}\label{table:xorro_approaches}
 	%\vspace{-5mm}
 	\begin{tabular}{ l|l }
 		Approach  & Description  \\
 		\hline\hline
 		count     & Add count aggregates with a modulo 2 operation  \\  
 		list,tree & Translate binary \XOR{} operators into rules forming list and tree structures \\
 		%            & (binary operators are arranged in list/tree)\\
 		countp    & Propagator simply counting truth literals on total assignments\\
 		up        & Propagator implementing unit propagation\\
 		gje       & Propagator implementing (non-incremental) Gauss-Jordan Elimination
 		
 	\end{tabular}
 %	\vspace{-7mm}
\end{table*}
 %
 Finally, we empirically evaluate the different approaches in view of their impact on solving performance,
 while varying the number and size of parity constraints compared against \clingo{} solving time.
 %
 The experiments show that \xorro{} scales depending on the combination of the number, density, and preprocessing of the parity constraints.
 %
 When increasing the number of high-density constraints as used in sampling (\XOR{}s with a size of half the program variables), we start to see that the solving time increases concerning \clingo.
 %
 Comparing to previous approaches, the eager counting, and the list (from the previous \xorro{} and \emph{harvey}), both stay behind for sampling purposes. Their scalability is subjected to the density of the parity constraints and preprocessing, and particularly, grounding becomes the bottleneck for the counting approach with aggregates.
\section{ApproxASP}

\textbf{ApproxMC Algorithm into ASP}

\begin{algorithm}[h]
\textbf{ApproxASP}($F$, $\epsilon$, $\delta$) \;
\SetAlgoLined
\KwResult{Approximate number of answer sets}
 $counter \gets 0$ ; $C \gets emptyList$\;
 $pivot \gets  2 \, \times $ \textbf{ComputeThreshold($\epsilon$)} \;
 $S \gets Xampler(F, pivot + 1)$ // {Solve pivot+1 AS}  \;
 \eIf{ $|S| \leq pivot$}
 { % Open If
 \textbf{return} $|S|$ \Comment*[r]{Exact count} 
 } % Close If
 { % Open Else
 $t \gets $ \textbf{ComputeIterCount($\delta$)} 
 
 \While{$counter < t$}{
 $c \gets $ \textbf{ApproxASPCore($F, pivot$)} \;
 $counter \gets counter + 1$ \;
 \If{$c \neq \bot$}{\textbf{AddToList($C,c$)}}
 } % Close While
 } % Close else
 $finalCount \gets $ \textbf{FindMedian($C$)} \;
 \textbf{return} $finalCount$
 
 \caption{ApproxMC into ASP}
\end{algorithm}


\begin{algorithm}
\textbf{return} $\lceil 3 e^{1/2} (1 + \frac{1}{\epsilon})^2 \rceil$
\caption{ComputeThreshold($\epsilon$)}
\end{algorithm}

\begin{algorithm}
\textbf{return} $\lceil 35 \log_2 (3/\delta) \rceil$
\caption{ComputeIterCount($\delta$)}
\end{algorithm}

\begin{algorithm}
/* Assume $z_1 , ..., z_n $ are the varaibles of $F$ */ \\
$ l \gets \lfloor \log_2 (pivot) - 1 \rfloor$ \\
$ i \gets l - 1$ \\

 \While{($1 \leq |S| \leq pivot $) or ($i = n$)}{
 $ i \gets i + 1$ \\
 Choose $h$ at random from $H_{xor}(n, i - l, 3)$ \\
 Choose $\alpha$ at random from $\{0,1\}^{i-l}$ \\
 $S \gets Xampler(F \land (h(z_1, ..., z_n)=\alpha, pivot + 1)$ %\Comment*[r]{Solve $P \land XOR$}  
 }
 
 \eIf{ $|S|$ $> pivot$ or $|S|$ = 0}
 {\textbf{return} $\bot$}
 {\textbf{return} $|S|$ $\cdot 2^{i-l}$}
 
 

\caption{ApproxASPCore($F, pivot$)}
\end{algorithm}




	\item Ankit: 
	    \begin{enumerate}
	    	\item Adapt the basic \approxmc algorithm to \asp.
	    	\item Collect all the relevant literature on the Counting Answer sets. Approx or Exact. 
	    \end{enumerate}
\end{enumerate}

\section{Introduction} \label{sec:introduction}
 Parity constraints constitute the foundations of reasoning modes like sampling or (approximate) model counting \cite{DBLP:journals/corr/abs-1806-02239}, as well as circuit verification and cryptography \cite{DBLP:conf/laitinen2014extending}.
 %
 With most of their applications in the neighboring area of Satisfiability Testing (SAT) \cite{DBLP:journals/corr/abs-1806-02239},
 almost no attention has so far been paid to their integration into Answer Set Programming (ASP).%~\cite{kalepesc16a}.
 
 Previous efforts represented parity constraints into ASP in three ways,
 via the \texttt{\#count} aggregate coupled with a modulo-two operation as used or sampling in the initial prototype of \xorro{} from 2009\footnote{https://sourceforge.net/p/potassco/code/HEAD/tree/branches/xorro},
 as lists, as shown in \emph{harvey}~\cite{DBLP:conf/lpnmr/GresslerOT17},
 and as the (discontinued) aggregates \texttt{\#even} and \texttt{\#odd} from \gringo{} series 3 via meta-encodings.
 %
 Unlike these approaches, several SAT solvers feature rather sophisticated treatments of parity constraints.
 For instance, most popularly the award-winning solver \cryptominisat~\cite{DBLP:conf/sat/SoosNC09}, which pursues a hybrid approach,
 addressing parity constraints separately with Gauss-Jordan Elimination (GJE).
 
 To this end, we present the next generation of \xorro,~\cite{DBLP:conf/lpnmr/EverardoJKS19}
 implementing six alternatives to handle parity constraints into ASP,
 benefiting from the advanced interfaces of \clingo{},
 and the sophisticated solving techniques developed in SAT.
 %
 We propose two types of approaches, eager and lazy.~\footnote{Both eager and lazy follows the methodology from Satisfiability modulo theories.}
 The former relies on ASP encodings of parity constraints, and the latter uses theory propagators within \clingo{}’s Python interface~\cite{DBLP:conf/iclp/GebserKKOSW16}.
 
 To accommodate parity constraints in the input language,
 we rely on \clingo’s theory language extension~\cite{DBLP:conf/iclp/GebserKKOSW16}
 following the common syntax of \emph{aggregates}~\cite{DBLP:journals/tplp/GebserHKLS15}:
 
 %
 % ------------------------------------------------------------------------------
 \lstinputlisting[basicstyle=\small\ttfamily]{listings/xor_constraints_ex1.lp}
 % ------------------------------------------------------------------------------
 %
 That is, \xorro{} extends the input language of \clingo{} by
 aggregate names \texttt{\&even} and \texttt{\&odd} that are followed
 by a set, whose elements are \emph{terms} conditioned by
 conjunctions of literals separated by commas.%
 %
 \footnote{In turn, multiple conditional terms within an aggregate are
 	separated by semicolons.}
 %
 In the context of a choice rule \texttt{\set{p(1..3)}.}, the parity constraints shown above
 amounts to the \XOR{} operations: \\
 $p(1) \xor \bot$ and $p(2) \xor p(3) \xor \top$
 (where $\bot$ and $\top$ stand for the Boolean constants true and false, respectively)
 yielding the answer sets~\texttt{\set{p(1)}} and \texttt{\set{p(1),p(2),p(3)}}.
 
 
 %The semantics of aggregates formed with keywords \texttt{\&even} and
 %\texttt{\&odd} is defined by \emph{even} and \emph{odd} parity constraints,
 %respectively.
 %In this implementation,
 Currently, these constraints are interpreted as directives,
 filtering answer sets that do not satisfy the parity constraint in question.
 \footnote{For now, parity constraints may not occur in the bodies nor the heads of rules.}
 %
 Hence, the first
 constraint filters out answer sets not containing the atom
 \texttt{p(1)}, while the second requires that either none or both of
 the atoms \texttt{p(2)} and \texttt{p(3)} are included.
 
 %As mentioned, \xorro{} handles parity constraints in six ways as shown in Table \ref{table:xorro_approaches},
 %implements different means to handle parity constraints as shown in Table \ref{table:xorro_approaches},
 %where the first three corresponds to the eager, and the last three to the lazy approaches.
 Table~ \ref{table:xorro_approaches}, shows the six implementations to handle parity constraints.
 The first three corresponds to the eager, and the last three to the lazy approaches. 
 
 \begin{table*}[t]
% 	\centering
\caption{\xorro{} approaches to handle parity constraints}\label{table:xorro_approaches}
 	%\vspace{-5mm}
 	\begin{tabular}{ l|l }
 		Approach  & Description  \\
 		\hline\hline
 		count     & Add count aggregates with a modulo 2 operation  \\  
 		list,tree & Translate binary \XOR{} operators into rules forming list and tree structures \\
 		%            & (binary operators are arranged in list/tree)\\
 		countp    & Propagator simply counting truth literals on total assignments\\
 		up        & Propagator implementing unit propagation\\
 		gje       & Propagator implementing (non-incremental) Gauss-Jordan Elimination
 		
 	\end{tabular}
 %	\vspace{-7mm}
\end{table*}
 %
 Finally, we empirically evaluate the different approaches in view of their impact on solving performance,
 while varying the number and size of parity constraints compared against \clingo{} solving time.
 %
 The experiments show that \xorro{} scales depending on the combination of the number, density, and preprocessing of the parity constraints.
 %
 When increasing the number of high-density constraints as used in sampling (\XOR{}s with a size of half the program variables), we start to see that the solving time increases concerning \clingo.
 %
 Comparing to previous approaches, the eager counting, and the list (from the previous \xorro{} and \emph{harvey}), both stay behind for sampling purposes. Their scalability is subjected to the density of the parity constraints and preprocessing, and particularly, grounding becomes the bottleneck for the counting approach with aggregates.
 
 %After running all experiments, we can see that the eager counting and list approaches stay behind when solving high-density constraints without any preprocessing (as in the previous xorro and harvey implementations).
 %
 %However, the eager counting and list approaches stay behind for sampling purposes when solving high-density constraints without any preprocessing as in the previous \xorro{} and \emph{harvey} implementations.
 
 %

ApproxMC~\cite{DBLP:conf/cp/ChakrabortyMV13}


\bibliographystyle{aaai}
\bibliography{Kr2020}

\end{document}
