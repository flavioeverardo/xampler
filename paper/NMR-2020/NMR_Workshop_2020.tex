%%%% kr-instructions.tex -- version 1.2 (27-Feb-2020)

\typeout{KR2020 Instructions for Authors}

% These are the instructions for authors for KR-20.

\documentclass{article}
\pdfpagewidth=8.5in
\pdfpageheight=11in

\usepackage{kr}

% Use the postscript times font!
\usepackage{times}
\usepackage{soul}
\usepackage{url}
\usepackage[hidelinks]{hyperref}
\usepackage[utf8]{inputenc}
\usepackage[small]{caption}
\usepackage{graphicx}
\usepackage{amsmath}
\usepackage{amssymb}
\usepackage{amsthm}
\usepackage{booktabs}
%\usepackage{algorithm}
%\usepackage{algorithmic}
\urlstyle{same}
\usepackage[ruled,vlined,linesnumbered]{algorithm2e}
% the following package is optional:
%\usepackage{latexsym}

% See https://www.overleaf.com/learn/latex/theorems_and_proofs
% for a nice explanation of how to define new theorems, but keep
% in mind that the amsthm package is already included in this
% template and that you must *not* alter the styling.
\newtheorem{example}{Example}
\newtheorem{theorem}{Theorem}

% added by Ankit
\usepackage{xspace}
\def\approxmc{\textsf{ApproxMC}\xspace}
\def\approxasc{\textsf{ApproxASC}\xspace}
\def\asp{\textsf{\#ASP}\xspace}
\def\SAT{\textsf{SAT}\xspace}

% complexity stuff

\newcommand{\bigO}[1]{\ensuremath{{\mathcal O}(#1)}}
\newcommand{\Nat}{\mathbb{N}} %
\newcommand{\CCard}[1]{\|#1\|}
\newcommand{\Card}[1]{\left|#1\right|}
\renewcommand{\P}{\ensuremath{\textsc{P}}\xspace}
\newcommand{\NP}{\ensuremath{\textsc{NP}}\xspace}
\newcommand{\coNP}{\ensuremath{\text{co-}\textsc{NP}}\xspace}
\newcommand{\SIGMA}[2]{\ensuremath{\Sigma_{\textrm{#1}}^{\textrm{#2}}}}
\newcommand{\WWW}{\ensuremath{\mathcal{W}}}%
\newcommand{\mtext}[1]{\ensuremath{\mathcal{#1}}}
\newcommand{\cnt}[0]{\ensuremath{\#}}
\newcommand{\cntc}[0]{\ensuremath{\cnt\cdot}}



\newtheorem{definition}{Definition}

% Copied from Flavio draft
\usepackage{url}
\usepackage{microtype}
\usepackage{hyperref}
\usepackage{float}
\usepackage{listings}

\usepackage{todonotes}
\setlength{\marginparwidth}{1.4cm}
%\usepackage[ruled,vlined]{algorithm2e}
%\SetKwComment{Comment}{$\triangleright$\ }{}

\newcommand{\XOR}{\textsc{xor}} % just an idea

%%%%%%%%% Added By Ankit
\newcommand{\fff}{$P$\xspace} 
\newcommand{\mc}{$mc$\xspace} 
\newcommand{\ep}{$\epsilon$\xspace} 
\newcommand{\del}{$\delta$\xspace} 
\def\appasp{\normalfont \textsf{ApproxASP}\xspace}
\def\countas{\normalfont \textsf{CountAS}\xspace}

\newcommand{\sol}{$Sol(P)$\xspace} 
\newcommand{\hash}{$\mathcal{H}$} 
\def\approxmc{\textsf{ApproxMC}\xspace}
% - systems ----------------------------------------------------------------------
% >>> NOT USED in BOOK <<<
\newcommand{\sysfont}{\textit}
\newcommand{\acthex}{\sysfont{acthex}}
\newcommand{\adsolver}{\sysfont{adsolver}}
\newcommand{\anthem}{\sysfont{anthem}}
\newcommand{\asparagus}{\sysfont{asparagus}}
\newcommand{\aspartame}{\sysfont{aspartame}}
\newcommand{\aspic}{\sysfont{aspic}}
\newcommand{\aspcud}{\sysfont{aspcud}}
\newcommand{\aspmt}{\sysfont{aspmt}}
\newcommand{\asprilo}{\sysfont{asprilo}}
\newcommand{\asprin}{\sysfont{asprin}}
\newcommand{\assat}{\sysfont{assat}}
\newcommand{\berkmin}{\sysfont{berkmin}}
\newcommand{\chuffed}{\sysfont{chuffed}}
\newcommand{\chasp}{\sysfont{chasp}}
\newcommand{\claspD}{\sysfont{claspD}}
\newcommand{\claspar}{\sysfont{claspar}}
\newcommand{\claspfolio}{\sysfont{claspfolio}}
\newcommand{\clasp}{\sysfont{clasp}}
\newcommand{\claspre}{\sysfont{claspre}}
\newcommand{\clingcon}{\sysfont{clingcon}}
\newcommand{\clingo}{\sysfont{clingo}}
\newcommand{\clingod}[1]{\clingo\textnormal{[}\textsc{#1}\textnormal{]}}
\newcommand{\cmodels}{\sysfont{cmodels}}
\newcommand{\coala}{\sysfont{coala}}
\newcommand{\cplex}{\sysfont{cplex}}
\newcommand{\dingo}{\sysfont{dingo}}
\newcommand{\dflat}{\sysfont{dflat}}
\newcommand{\dlvhex}{\sysfont{dlvhex}}
\newcommand{\dlv}{\sysfont{dlv}}
\newcommand{\ezcsp}{\sysfont{ezcsp}}
\newcommand{\ezsmt}{\sysfont{ezsmt}}
\newcommand{\embasp}{\sysfont{embasp}}
\newcommand{\fastdownward}{\sysfont{fastdownward}}
\newcommand{\ftolp}{\sysfont{f2lp}}
\newcommand{\gasp}{\sysfont{gasp}}
\newcommand{\gfd}{\sysfont{g12fd}}
\newcommand{\gecode}{\sysfont{gecode}}
\newcommand{\gidl}{\sysfont{gidl}}
\newcommand{\gnt}{\sysfont{gnt}}
\newcommand{\gringo}{\sysfont{gringo}}
\newcommand{\harvey}{\sysfont{harvey}}
\newcommand{\iclingo}{\sysfont{iclingo}}
\newcommand{\idp}{\sysfont{idp}}
\newcommand{\inca}{\sysfont{inca}}
\newcommand{\jdlv}{\sysfont{jdlv}}
\newcommand{\lctocasp}{\sysfont{lc2casp}}
\newcommand{\lparse}{\sysfont{lparse}}
\newcommand{\lpsolve}{\sysfont{lpsolve}}
\newcommand{\lptodiff}{\sysfont{lp2diff}}
\newcommand{\lptosat}{\sysfont{lp2sat}}
\newcommand{\mchaff}{\sysfont{mchaff}}
\newcommand{\measp}{\sysfont{measp}}
\newcommand{\metasp}{\sysfont{metasp}}
\newcommand{\mingo}{\sysfont{mingo}}
\newcommand{\minisat}{\sysfont{minisat}}
\newcommand{\minisatid}{\sysfont{minisatid}}
\newcommand{\minizinc}{\sysfont{minizinc}}
\newcommand{\mznfzn}{\sysfont{mzn2fzn}}
\newcommand{\nomorepp}{\sysfont{nomore++}}
\newcommand{\oclingo}{\sysfont{oclingo}}
\newcommand{\omiga}{\sysfont{omiga}}
\newcommand{\piclasp}{\sysfont{piclasp}}
\newcommand{\picosat}{\sysfont{picosat}}
\newcommand{\picat}{\sysfont{picat}}
\newcommand{\picatsat}{\sysfont{picatsat}}
\newcommand{\plasp}{\sysfont{plasp}}
\newcommand{\quontroller}{\sysfont{quontroller}}
\newcommand{\reify}{\sysfont{reify}}
\newcommand{\rosoclingo}{\sysfont{rosoclingo}}
\newcommand{\sag}{\sysfont{sag}}
\newcommand{\satz}{\sysfont{satz}}
\newcommand{\siege}{\sysfont{siege}}
\newcommand{\smac}{\sysfont{smac}}
\newcommand{\smodelscc}{\sysfont{smodels$_{\!cc}$}}
\newcommand{\smodelsr}{\sysfont{smodels}$_r$}
\newcommand{\smodels}{\sysfont{smodels}}
\newcommand{\sugar}{\sysfont{sugar}}
\newcommand{\teaspoon}{\sysfont{teaspoon}}
\newcommand{\tel}{\sysfont{tel}}
\newcommand{\telingo}{\sysfont{telingo}}
\newcommand{\unclasp}{\sysfont{unclasp}}
\newcommand{\wasp}{\sysfont{wasp}}
\newcommand{\xorro}{\sysfont{xorro}}
\newcommand{\xampler}{\sysfont{xampler}}
\newcommand{\zchaff}{\sysfont{zchaff}}
\newcommand{\zzz}{\sysfont{z3}}
\newcommand{\cryptominisat}{\sysfont{crypto-minisat}}

\newcommand{\clingoM}[1]{\clingo{\small\textnormal{[}\textsc{#1}\textnormal{]}}}
\newcommand{\ASPm}[1]{ASP\raisebox{.7pt}{[\textsc{#1}]}}

\newcommand{\flatzinc}{\sysfont{FlatZinc}}
\newcommand{\aspif}{\sysfont{aspif}}

\newcommand{\python}{Python}
\newcommand{\lua}{Lua}
\newcommand{\cpp}{C++}
\newcommand{\C}{C}
\newcommand{\java}{Java}
\newcommand{\haskell}{Haskell}

\newcommand{\el}[1]{\vspace{#1\baselineskip}}

% Some notational macros

\newcommand{\rg}[3]{{#1}{#2}\,\ldots{#2}\,{#3}}
\newcommand{\xor}{\oplus}
\newcommand{\true}{\top}
\newcommand{\false}{\bot}
\newcommand{\naf}{\mathit{not}\,}
\newcommand{\set}[1]{\{#1\}}

% Tomi's specialties (not needed in the end)

\newcommand{\TODO}[1]{%
	\vspace{1\baselineskip}\noindent\hrulefill\hspace{1em}{#1}%
	\hspace{1em}\hrulefill\vspace{1\baselineskip}}

\newcommand{\DONE}{%
	\vspace{1\baselineskip}\noindent\hrule\vspace{1\baselineskip}}

% Following comment is from ijcai97-submit.tex:
% The preparation of these files was supported by Schlumberger Palo Alto
% Research, AT\&T Bell Laboratories, and Morgan Kaufmann Publishers.
% Shirley Jowell, of Morgan Kaufmann Publishers, and Peter F.
% Patel-Schneider, of AT\&T Bell Laboratories collaborated on their
% preparation.

% These instructions can be modified and used in other conferences as long
% as credit to the authors and supporting agencies is retained, this notice
% is not changed, and further modification or reuse is not restricted.
% Neither Shirley Jowell nor Peter F. Patel-Schneider can be listed as
% contacts for providing assistance without their prior permission.

% To use for other conferences, change references to files and the
% conference appropriate and use other authors, contacts, publishers, and
% organizations.
% Also change the deadline and address for returning papers and the length and
% page charge instructions.
% Put where the files are available in the appropriate places.

\title{An Approximate Model Counter for ASP}

% Single author syntax
\iffalse % (remove the multiple-author syntax below and \iffalse ... \fi here)
\author{%
    Author name
    \affiliations
    Affiliation
    \emails
    email@example.com    % email
}
\fi
% Multiple author syntax
\author{%
Flavio Everardo$^1$\and
Markus Hecher$^2$\and
Ankit Shukla$^{3}$
\affiliations
$^1$ University of Potsdam, Germany\\
$^2$ TU Wien, Vienna, Austria\\
$^3$ JKU, Linz, Austria\\
\emails
flavio.everardo@cs.uni-potsdam.de,
mhecher@gmail.com,
ankit.shukla@jku.at
}

\begin{document}

\maketitle

\begin{abstract}
%Answer Set Programming (ASP) is a declarative framework that is well-suited for problems in KR, AI and other areas as well as plenty of practical applications.
%While modern ASP solvers not only compute one solution (answer set), but actually support different reasoning modes, the problem of counting answer sets has not been subject to intense studies yet.
Answer Set Programming (ASP) is a declarative framework that is well-suited for problems in KR, AI, and other areas with plenty of practical applications.
While modern ASP solvers not only compute one solution (answer set) but actually support different (reasoning) problems, the problem of counting answer sets has not been subject to intense studies.
This is in contrast to propositional satisfiability (SAT), where several applications and problems related to quantitative reasoning trace back to model counting. However, due to high computational complexity and depending on the actual application, approximate counting might be sufficient. Indeed, there are plenty of applications, where approximate counting for SAT is well-suited. This work deals with lifting \xorro{}, which builds on top of an ASP solver and allows XOR constraints, towards approximate model counting for ASP. We present preliminary experiments using the resulting approximate counter for ASP. % and also show preliminary experiments for several problems.
While we do not have specific guarantees in terms of accuracy, our preliminary results look promising.
\end{abstract}

\setcounter{section}{-1}
\section{TODOS} \label{sec:todos}
\begin{enumerate}
	\item Focus on extending \emph{xorro} to support approximate ASP counting rather than selling it as the first ApproxASP counter (although we should point out that this is the first such attempt).
	\item This is an ongoing work. 
	\item Narrate that this initial work could be helpful to generate interest in approx asp counting.
	\item Questions will be asked if shoplifting the ApproxMC is fair and what else we need to ensure the theoretical bounds. At least some fundamental understanding should be outlined based on models of SAT and ASP.  
\end{enumerate}

\section{Introduction} \label{sec:introduction}
%
Answer Set Programming (ASP)~\cite{Lifschitz99,BrewkaEiterTruszczynski11,GebserKaminskiKaufmannSchaub12} is a
problem modeling and solving framework that is well-known in the area
of knowledge representation and reasoning and artificial intelligence.
%
This framework has been practically applied for several
problems~\cite{BalducciniGelfondNogueira06a,NiemelaSimonsSoininen99,NogueiraBalducciniGelfond01a,GuziolowskiEtAl13a,SchaubWoltran18}.
%

Recently, there has been growing interest for counting solutions of problems.
%
Indeed, counting solutions is a well-known task not only in mathematics and computer
science, but also in other
areas~\cite{ChakrabortyMeelVardi16a,DomshlakHoffmann07a,GomesKautzSabharwalSelman08a,SangBeameKautz05a}.
%
%
%
Examples cover also applications in machine learning and probabilistic inference~\cite{ChaviraDarwiche08a}.
%
%
In terms of computational complexity, counting has been well-studied since the
late
70s~\cite{DurandHermannKolaitis05,HemaspaandraVollmer95a,Valiant79,Valiant79b}.
%
There are also results for counting involving projection, where one wants to
count only with respect to a given set of projection atoms,
which has been established for logic~\cite{AzizChuMuise15a,CapelliMengel19,FichteEtAl18,LagniezMarquis19a,GuptaSharmaRoy19a,SharmaRoySoos19a},
reliability estimation~\cite{MeelEtAl17a} as well as
answer set programming~\cite{GebserKaufmannSchaub09a,Aziz15a,FichteHecher19}.
%

Given that in general counting answer sets is rather hard, namely ${\#\cdot\text{coNP}}$-complete~\cite{FichteEtAl17,DurandHermannKolaitis05}, which further increases
to ${\#\cdot\Sigma_2^P}$-completeness~\cite{FichteHecher19} if counting with respect to a projection,
a different approach than exact counting seems to be required in practice.
%
Indeed, such approaches were successful for propositional logic (SAT), which is~$\cntc\P$-complete and where reasoning modes like sampling (near-uniform generation)
or (approximate) model counting~\cite{gosase07a,chmeva13a,ChakrabortyMV13,SharmaRoySoos19a}
were studied.
%
For this purpose, so-called parity constraints  (\XOR{} constraints) are used specifically to partition the search space in parts that preferably are of roughly the same size.

Parity constraints have been recently accommodated in ASP
as the fundamental part of the system \xorro{}~\cite{DBLP:conf/lpnmr/EverardoJKS19}.
%
With different solving approaches over parity constraints in \xorro{}, these constraints amount to the classical \XOR{} operator following the aggregates-like syntax by means of theory atoms.
They are interpreted as directives, solved on top of an ASP program acting as answer set filters that do not satisfy the parity constraint in question.
%
%Currently, these parity constraints may not occur in the bodies nor the heads of rules.



%Parity constraints constitute the foundations of reasoning modes like sampling or (approximate) model counting \cite{DBLP:journals/corr/abs-1806-02239}, as well as circuit verification and cryptography \cite{DBLP:conf/laitinen2014extending}.
%
With most of the applications of \XOR{} constraints in the neighboring area of SAT~\cite{DBLP:journals/corr/abs-1806-02239},
only few attention has been paid to treat parity constraints as well as reasoning modes like sampling or approximate model counting for ASP. %~\cite{kalepesc16a}.
%
To this end, we present an extension of \xorro{} towards approximate answer set counting following the work from~\cite{ChakrabortyMV13}
benefiting from the advanced interfaces of ASP solver \clingo{}~\cite{DBLP:conf/iclp/GebserKKOSW16},
and the sophisticated solving techniques developed in SAT~(e.g. the award-winning solver \cryptominisat~\cite{DBLP:conf/sat/SoosNC09}).
%
While we do not yet have theoretical guarantees in terms of accuracy of our approach in general,
the results look promising and we hope that this will foster further research in approximate answer set counting.
%
%This paper is structured in ...

\section{Preliminaries}

\smallskip\noindent\textbf{Computational Complexity.}
%
%
We assume familiarity with standard notions in
  computational complexity~\cite{Papadimitriou94} and use counting complexity classes
  as defined by Durand, Hermann and
  Kolaitis~\cite{DurandHermannKolaitis05}.
%
%}
%
%
% which are different from the counting complexity classes introduced
% Valiant~\cite{Valiant79,Valiant79b}, in particular, they allow for a
% more fine-grained distinction of counting complexity. % beyond \#P.
%
% and follow the notation by Hesaspaandra and
% Vollmer~\cite{HemaspaandraVollmer95a} for counting complexity classes,
% which are different from the counting complexity classes introduced
% Valiant~\cite{Valiant79,Valiant79b}, in particular, they allow for a
% more fine-grained distinction of counting complexity. % beyond \#P.
%
%For parameterized complexity, we refer to standard
%texts~\cite{CyganEtAl15}. 
%
% ,DowneyFellows13,FlumGrohe06,Niedermeier06}.
%
%We recall some basic notions.
%
Let $\Sigma$ and $\Sigma'$ be finite alphabets, $I \in \Sigma^*$ an \emph{instance}, and $\CCard{I}$
%%
 denote the size of~$I$. %
%\longversion{ denote its size.}
% Let $L\subseteq \Sigma^*$ and $L' \subseteq \Sigma'^*$ be
% decision problems.
%
%Let $L \subseteq \Sigma^* \times \Nat$ and
%$L' \subseteq {\Sigma'}^*\times \Nat$ be two parameterized problems. 
%An \emph{fpt-reduction} $r$ from $L$ to $L'$ is a many-to-one reduction
%from $\Sigma^*\times \Nat$ to ${\Sigma'}^*\times \Nat$ such that for all
%$I \in \Sigma^*$ we have $(I,k) \in L$ if and only if
%$r(I,k)=(I',k')\in L'$ such that $k' \leq g(k)$ for a fixed computable
%function $g: \Nat \rightarrow \Nat$, and there is a computable function
%$f$ and a constant $c$ such that $r$ is computable in time
%$O(f(k)\CCard{I}^c)$. If additionally~$g$ is linear,
%then~$r$ is referred to as~\emph{fptl-reduction}.
%~\cite{FlumGrohe06}.
%
%
%
A \emph{witness function} %is a
%function
 $\mathcal{W}: \Sigma^* \rightarrow 2^{{\Sigma'}^*}$ %that
maps an instance~$I \in \Sigma^*$ to its \emph{witnesses}. 
%a finite subset of~${\Sigma'}^*$. %We call the set~$\WWW(I)$ the \emph{witnesses}. 
A
\emph{counting
  problem}~$L: \Sigma^* \rightarrow \Nat_0$ is a
function that maps a given instance~$I \in \Sigma^*$ to the cardinality of its
witnesses~$\Card{\WWW(I)}$.
%
%
Let $\mtext{C}$ be a decision complexity class,~e.g., \P. Then,
$\cntc\mtext{C}$ denotes the class of all counting problems whose
witness function~$\WWW$ satisfies (i)~there is a
function~$f: \Nat_0 \rightarrow \Nat_0$ such that for every
instance~$I \in \Sigma^*$ and every $W \in \WWW(I)$ we have
$\Card{W} \leq f(\CCard{I})$ and $f$ is computable in
time~$\bigO{\CCard{I}^c}$ for some constant~$c$ and (ii)~for every
instance~$I \in \Sigma^*$ decision problem~$\WWW(I)$ is in the complexity class~$\mtext{C}$.
% ``given~$x$ and~$y$, is $y \in w(x)$?''
Then, $\cntc\P$ is the complexity class consisting of all counting
problems associated with decision problems in \NP.
%
% unused
%
% We say that a problem~$L$ is \emph{$\mtext{C}\hy \text{hard}$} if
% there is a polynomial-time reduction or parsimonious reduction,
% respectively, for every problem~$L' \in \mtext{C}$ to $L$. If in
% addition $L \in \mtext{C}$, then $L$ is
% \emph{$\mtext{C}\hy \text{complete}$}.  For instance, a decision
% problem is \NP-complete if it belongs to \NP and all decision
% problems in \NP have polynomial-time reductions to it.



\smallskip
\noindent\textbf{Answer Set Programming (ASP).} %
% ASP is a declarative modeling and problem solving framework that
% combines techniques of knowledge representation and database
% theory. Two of the main advantages of ASP are its
% expressiveness~\cite{BrewkaEiterTruszczynski11} and, when using
% non-ground programs, its advanced declarative problem modelling
% capability. Prior to solving, non-ground programs are usually compiled
% into ground ones by a grounder.  For particular non-ground programs,
% treewidth guarantees are known~\cite{BliemEtAl17a}.
% In this paper, we restrict ourselves to ground ASP programs.  For a
% comprehensive introduction, we refer to introductory
% literature~\cite{BrewkaEiterTruszczynski11,JanhunenNiemela16a}.
%
We assume familiarity with propositional satisfiability (\SAT)~\cite{KleineBuningLettman99}
and follow standard definitions of propositional ASP~\cite{BrewkaEiterTruszczynski11}.
%For comprehensive foundations, we refer to introductory
%literature~\cite{BrewkaEiterTruszczynski11,JanhunenNiemela16a}.
%
Let $m$, $n$, $\ell$ be non-negative integers such that
$m \leq n \leq \ell$, $a_1$, $\ldots$, $a_\ell$ be distinct propositional
atoms. Moreover, we refer by \emph{literal} to an atom or the negation
thereof.
%, and
%$l \in \{a_1, \neg a_1\}$ being a \emph{literal},~i.e., an atom or the negation thereof.
%
A \emph{(logic) program}~$\Pi$ is a set of \emph{rules} of the form
%
%\begin{align*}
%\smallskip
%\hfill
%\vspace{-0.75em}
\(
a_1\vee \cdots \vee a_m \leftarrow a_{m+1}, \ldots, a_{n}, \neg
a_{n+1}, \ldots, \neg a_\ell.
%
\)
%\hfill
%\smallskip
%
%\end{align*}
%
%
% \noindent Intuitively this requires at least one atom of
% $a_1, \ldots, a_\ell$ to be true if all atoms
% $a_{\ell+1}, \ldots, a_m$ are true and there is no evidence that any
% atoms of $a_{m+1}, \ldots, a_n$ are true. The set of
% \emph{well-defined programs} among atoms~$X$ is given by~$prog(X)$.
%
%
%
%Let $\at(\Pi) \eqdef \SB
%\at(r) \SM r \in \Pi \SE$ be the atoms occuring in a program~$\Pi$,
%
%Therefore, let $\choice(\Pi), \disj(\Pi),
%\opt(\Pi)$ and $\weight(\Pi)$ denote the set of all choice, disjunctive,
%optimization and weight rules in a program~$\Pi$, respectively.
%
%
%\noindent
For a rule~$r$, we let $H_r = \{a_1, \ldots, a_m\}$,
$B^+_r = \{a_{m+1}, \ldots, a_{n}\}$, and
$B^-_r = \{a_{n+1}, \ldots, a_\ell\}$.
%
% We call $H_r$ the \emph{head}, $B^+_r$ the \emph{positive body}, and
% $B^-(r)$ the \emph{negative body} of~$r$.
%
% , and $B_r \eqdef B^+_r \cup \SB \neg b \SM b \in B^-_r \SE$ the
% \emph{body} of~$r$.\nb{Do we really need the body?}
%
We denote the sets of \emph{atoms} occurring in a rule~$r$ or in a
program~$\Pi$ by $\text{at}(r) = H_r \cup B^+_r \cup B^-_r$ and
$\text{at}(\Pi)= \bigcup_{r\in\Pi} \text{at}(r)$.
%
%For a set~$X\subseteq\at(\Pi)$ of atoms, we let~$\overline{X}\eqdef \{\neg x\mid x\in X\}$.
%
%Let $\Pi$ be a program.
%
%A program~$\Pi'$ is a \emph{sub-program of~$\Pi$} if~$\Pi'\subseteq\Pi$.
%A program~$\Pi'$ is a \emph{sub-program of~$\Pi$}, if symbols
%$\Pi' \subseteq \Pi$, if for each rule~$r' \in P'$ there is
%%exactly one
%a rule~$r \in \Pi$ with $H_{r'} \subseteq H_r$,
%$B^+_{r'} \subseteq B^+_r$, and $B^-_{r'} \subseteq B^-_r$. \nb{why so complicated sub-programs?}
%
%Program~$\Pi$ is \emph{normal}, if $\Card{H_r} \leq 1$ for
%every~$r \in \Pi$.
%
%The \emph{positive dependency digraph}~$D_\Pi$ of $\Pi$ is the
%directed graph defined on the set~$\bigcup_{r\in \Pi}H_r \cup B^+_r$ of atoms, where there is a directed edge from vertex~$a$ to~$b$ iff there is a rule~$r \in \Pi$ with $a\in B^+_r$ and~$b\in H_r$.
%
%
% The class~\HCF contains all head-cycle-free programs.\nb{do we need
% the class?}

An \emph{interpretation} $I$ is a set of atoms. $I$ \emph{satisfies} a
rule~$r$ if $(H_r\,\cup\, B^-_r) \,\cap\, I \neq \emptyset$ or
$B^+_r \setminus I \neq \emptyset$.  $I$ is a \emph{model} of $\Pi$
if it satisfies all rules of~$\Pi$, in symbols $I \models \Pi$. %, if $I$ satisfies every rule~$r \in \Pi$.
%
%For brevity, we view propositional formulas
%as sets of clauses that need to be satisfied, and
%use the notion of interpretations, models, and satisfiability analogously. %and
%satisfiability
%also in the context of satisfiability of
%propositional formulas.
%\nb{only used once}
%
The \emph{Gelfond-Lifschitz
  (GL) reduct} of~$\Pi$ under~$I$ is the program~$\Pi^I$ obtained
from $\Pi$ by first removing all rules~$r$ with
$B^-_r\cap I\neq \emptyset$ and then removing all~$\neg z$ where
$z \in B^-_r$ from every remaining
rule~$r$~\cite{GelfondLifschitz91}. %
%
$I$ is an \emph{answer set} of a program~$\Pi$ if $I$ is a minimal
model of~$\Pi^I$. %
%
%Deciding whether a disjunctive program has an answer set is
%\SIGMA{2}{P}-complete~\cite{EiterGottlob95}.
The problem of deciding whether an ASP program has an answer set is called
\emph{consistency}, which is $\Sigma_2^P$-complete~\cite{EiterGottlob95}.
%
If no rule uses disjunction, the complexity drops to
\text{NP}-complete~\cite{BidoitFroidevaux91,MarekTruszczynski91}.

%The problem~$\asp$ asks for a given program~$\Pi$ to compute the number of
%answer sets of~$\Pi$.
%In general we have that problem~$\asp$ is
%$\cntc\coNP$-complete~\cite{FichteEtAl17}.

\begin{example}
Assume a graph~$G$ consisting of vertices~$V=a,b,c,d$
and edges~$E=\{\{a,b\}, \{a,c\}, \{b,c\}, \{c,d\}\}$.
Then, the \emph{vertex cover problem} asks for a set~$S\subseteq V$ of vertices
such that for each edge~$e\in E$ we have that~$S\cap e\neq \emptyset$.
An extension is the \emph{subset-minimal vertex cover problem}, where we ask
only for sets~$S$, where no subset~$S'\subsetneq S$ is a vertex cover of~$G$.
We elegantly encode the computation of subset-minimal vertex covers into an ASP program~$\Pi$ as follows:
For each edge~$\{u,v\}\in E$, program~$\Pi$ contains the rules~$u \vee v \leftarrow$.
Observe that the resulting program~$\Pi$ indeed precisly characterizes the subset-minimal vertex covers of~$G$, which are~$\{a,c\}$, $\{b,c\}$, and $\{a,b,d\}$.
\end{example}

\noindent\textbf{Answer Set Counting (\asp).}
The problem~$\asp$ asks for a given program~$\Pi$ to compute the number of
answer sets of~$\Pi$.
In general we have that problem~$\asp$ is
$\cntc\coNP$-complete~\cite{FichteEtAl17}.
If we restrict the problem~$\asp$ to normal programs without disjunction,
the complexity drops to~$\cntc\P$-completeness,
which is easy to see via standard reductions
that preserve the number of solutions from and to propositional satisfiability (\SAT), see, e.g.,~\cite{Janhunen06}.

\begin{example}
Recall the previous example, graph~$G$ and example program~$\Pi$.
Since~$G$ has~$3$ subset-minimal vertix covers, the solution to problem~$\asp$
for given program~$\Pi$ is~$3$.
\end{example}


\section{Parity Constraints, \emph{xorro}, and Search Space Partition} \label{sec:parity}

Towards the definition of parity aggregates, let $\true$ and $\false$
stand for the Boolean constants \emph{true} and \emph{false},
respectively, and a literal is an atom $a$ and its negation $\neg a$.
\footnote{Recall that in the context of ASP, we understand the negation $\neg$ as the default negation.}
%
Given atoms $a_1$ and $a_2$, the \emph{exclusive or}
(\XOR{} for short) of $a_1$ and $a_2$ is denoted by $a_1\xor a_2$ and
it is satisfied if \emph{either} $a_1$ \emph{or} $a_2$ is true (but
not both).
Generalizing the idea for $n$ distinct atoms
$\rg{a_1}{,}{a_n}$, we obtain an $n$-ary \XOR{} constraint
$\rg{(((a_1\xor a_2}{)}{\xor\,a_n)}$ by multiple applications of
$\xor$. Since it is satisfied iff an odd number of atoms among
$\rg{a_1}{,}{a_n}$ are true,
we can simply refer it to as an \emph{odd parity constraint}
and it can be written simply as
$\rg{a_1}{\xor}{a_n}$ due to associativity.
%
Analogously, an \emph{even parity} (or \XOR{}) \emph{constraint} is defined by
$\rg{a_1}{\xor}{a_n}\xor\true$ as it is satisfied iff an even number
of atoms among $\rg{a_1}{,}{a_n}$ hold. Then, e.g.,
%
$a_1\xor a_2\xor\true$
%
is satisfied iff none or both of $a_1$ and $a_2$ hold.
%
Similarly, an \emph{even parity constraint} can be represented in terms of the \emph{odd} parity by an uneven
number of negated literals.
For instance, $\neg a_1\xor a_2$ is equivalent to $a_1\xor a_2\xor\true$ and pairs of negated literals cancel parity inversion, for example $\neg a_1\xor \neg a_2$ is equivalent to $a_1\xor a_2$
%Following~\cite{hanjia12a,laitinen14a} a \emph{parity constraint} is in a \emph{normal(or standard)-form} if all of its literals
%are positive and (pairs of) duplicates are removed.
%For example, the parity constraint $a_1\xor a_2\xor \neg a_3\xor a_2$ equals to $a_1\xor a_3\xor\true$.
Finally, \XOR{} constraints of forms $a\xor\false$ and $a\xor\true$
are called \emph{unary}.

To accommodate parity constraints in ASP's input language,
we rely on \clingo’s theory language extension~\cite{DBLP:conf/iclp/GebserKKOSW16}
following the common syntax of \emph{aggregates}~\cite{DBLP:journals/tplp/GebserHKLS15}:

\begin{verbatim}
&even{1:p(1);4: not p(4);5:p(5)}.
&odd{2:not p(2);5: not p(5);6:p(6)}.
&odd{X:p(X), X > 2}.
&even{X:p(X), X < 3}.
&odd{5:p(5)}.
\end{verbatim}

%
That is, \xorro{} extends the input language of \clingo{} by
aggregate names \texttt{\&even} and \texttt{\&odd} that are followed
by a set, whose elements are \emph{terms} conditioned by
conjunctions of literals separated by commas.%
%
\footnote{In turn, multiple conditional terms within an aggregate are
	separated by semicolons.}
%
From a search space of 64 answer sets in the context of the choice rule \texttt{\set{p(1..6)}.}, the parity constraints shown above
amounts to the \XOR{} operations:
$p(1) \xor p(4) \xor p(5)$, $p(2) \xor p(5) \xor p(6)$, $p(3) \xor p(4) \xor p(5) \xor p(6)$, $p(1) \xor p(2) \xor \top$, and $p(1) \xor \bot$
%(where $\bot$ and $\top$ stand for the Boolean constants true and false, respectively)
yield the answer sets~\texttt{\set{p(5)}} and \texttt{\set{p(1),p(2),p(4),p(5),p(6)}}.
This means that each parity constraint divides the search space (roughly) by half, and for this symmetric search space example with five parity constraints ($m=5$),
we have $2^m$ clusters, each with two answer sets.


%The semantics of aggregates formed with keywords \texttt{\&even} and
%\texttt{\&odd} is defined by \emph{even} and \emph{odd} parity constraints,
%respectively.
%In this implementation,
Currently, these constraints are interpreted as directives,
filtering answer sets that do not satisfy the parity constraint in question.
\footnote{Cannot occur either in the bodies or heads of rules.}
%
Hence, the first two
constraints hold for any combination of an uneven number of literals from \texttt{p(1), p(4), p(5)} and \texttt{p(2), p(5), p(6)} respectively.
The third constraint holds if an odd number of literals from \texttt{p(3)} to \texttt{p(6)} are true, while the fourth constraint requires that either none or both of
the atoms \texttt{p(1)} and \texttt{p(2)} are included.
The last parity constraint discards any answer set not containing the atom \texttt{p(5)}.

\xorro{} handles parity constraints in six different ways,
switching between ASP encodings of parity constraints (eager approaches), and the use of theory propagators within \clingo{}'s Python interface (lazy approaches).~\footnote{The distinction of eager and lazy approaches follows the methodology in Satisfiability modulo theories~\cite{baseseti09a}.}
%as shown in Table~\ref{table:xorro_approaches},
%where the first three corresponds to ASP encodings of parity constraints (eager approaches), and the last three to theory propagators within \clingo{}'s Python interface (lazy approaches).~\footnote{The distinction of eager and lazy approaches follows the methodology in Satisfiability modulo theories~\cite{baseseti09a}.}

%As mentioned, \xorro{} handles parity constraints in six ways as shown in Table \ref{table:xorro_approaches},
%implements different means to handle parity constraints as shown in Table \ref{table:xorro_approaches},
%where the first three corresponds to the eager, and the last three to the lazy approaches.
%Table~ \ref{table:xorro_approaches}, shows the six implementations to handle parity constraints.
%The first three corresponds to the eager, and the last three to the lazy approaches.



%\begin{table*}[t]
%	% 	\centering
%	\caption{\xorro{} approaches to handle parity constraints}\label{table:xorro_approaches}
%	%\vspace{-5mm}
%	\begin{tabular}{ l|l }
%		Approach  & Description  \\
%		\hline\hline
%		count     & Add count aggregates with a modulo 2 operation  \\
%		list,tree & Translate binary \XOR{} operators into rules forming list and tree structures \\
%		%            & (binary operators are arranged in list/tree)\\
%		countp    & Propagator simply counting truth literals on total assignments\\
%		up        & Propagator implementing unit propagation\\
%		gje       & Propagator implementing (non-incremental) Gauss-Jordan Elimination

%	\end{tabular}
%	%	\vspace{-7mm}
%\end{table*}
%

\section{Universal hashing and xor's} \label{sec:hashing}
Given an input ASP program \fff, the ASP counting problem (\asp) computes the total number of solutions, denoted by \sol of the program \fff.
%
One approach is to count all the solutions by enumeration (almost), known as exact counting.
% 
The exact counting gives precise results but the complexity of exact solution count for large problem instances hinders the application of exact model counting.
%
There are ongoing works \cite{FichteEtAl17} in improving exact model counters, but a broad set of practical applications has remained beyond the reach. 
%


In applications of model counting like probabilistic reasoning, planning with uncertainty etc, it may be sufficient to approximate the solution count and avoid the overhead of exact model counting.
%
An approximate counting tries to approximately compute the number of solutions by using a probabilistic algorithm $\appasp(F, \epsilon, \delta)$.
%
The algorithm takes a program \fff with a tolerance $\epsilon > 0$ and a confidence $0 < \delta \leq 1$ as an input.
%
The output is an estimate \mc based on the parameter \ep and \del.
%
Proving theoretical bounds and adapting the algorithm for the same is an ongoing work.

The central idea of the approximate model counting approach is the use of hash functions to partition the \sol for a program \fff, into roughly equal small cells.
%
Then pick a random cell and scale it by the number of cells to obtain an \ep-approximate estimate of the model count.
%
Note that apriori we do not know the distribution of the solution set, \sol.
%
We have to perform a good hashing without the knowledge of the distribution of solutions. i.e., partition the $Sol(\phi)$ into cells with a roughly equal number of solutions.
%
%This difficulty is handled by the use of \textbf{universal hashing}~\cite{DBLP:conf/stoc/CarterW77}.
%
%%
%%To partition the $Sol(\phi)$ into roughly equal small cells without knowing the distribution of the solution we use universal hashing.
%The key idea of universal hashing is to select a hash function randomly from a family of hash functions with low probability of collision between any two keys. 
%%
%By choosing a random hash function from a family, we randomize over arbitrary input distribution. This guarantees a low number of collisions in expectation irrespective of how data is chosen.
%
The difficulty is resolved by the use of \textbf{universal hashing}~\cite{DBLP:conf/stoc/CarterW77}.

The universal hashing selects a hash function randomly from a family of hash functions with a low probability of collision between any two keys. 
%
By choosing a random hash function from a family, we randomize over arbitrary input distribution. This guarantees a low number of collisions in expectation irrespective of how data is chosen.

\begin{definition}
A family of functions $\mathcal{H}=\{h: U \to [m]\}$  is called a universal family if,
\[ \forall x,y\in U,~x\neq y:~~\Pr _{h\in H}[h(x)=h(y)]\leq {\frac {1}{m}} \]
\end{definition}

if the hash function $h$ is drawn randomly from \hash, the probability at which any two keys of the universe $U$ will collide is at most $1/m$.
%
This probability of collision is what we generally expect if the hash function assigned truly random hash codes $h(x)$ to every key $x \in U$.

For model counting with high guarantees this is not enough, we not only need every input to be hash uniformly but also hashed independently. We need a family of $k$-wise independent hash functions~\cite{DBLP:conf/focs/WegmanC79}.
%
%$H(n, m, k)$:
%Need Family of $k$-wise independent hash functions mapping $\{0, 1\}^{U}$
% $\to$ $\{0, 1\}^{m}$ such that every $k$ elements are mapped independently.
%Technical definition generally used for the $k$-wise independent:
% i.e.,  
A random hash function $h$ is $k$-independent if for all choices of distinct $x_1$,…,$x_k$ the values $h(x_1)$,…,$h(x_k)$ are independent.
\begin{definition}
	Let $\mathcal{H}=\{h: U \to [m]\}$ be a family of hash functions. For any distinct $x_1$, . . . , $x_k \in U$ and any $y_1$, ..., $y_k \in [m]$ we have the following guarantee.

	$~~\Pr_{h\in H}
	[h(x_1) = y_1 \land ... \land h(x_k) = y_k] = \frac {1}
	{m^
		k}$
\end{definition}

Intuitively a family $\mathcal{H} = {h : U \to [m]}$ of hash functions
is $k$-wise independent if for any distinct keys $x_1, ..., x_k \in U$, the hash codes $h(x_1), ..., h(x_k)$ are independent random variables and for any fixed $x$, $h(x)$ is uniformly distributed in $[m]$.
%\end{itemize}

The canonical construction of a $k$-wise independent family is based on polynomials
of degree $k$ - $1$. Let $p \geq |U|$ be prime. Picking random $a_0,...,a_{k - 1} \in \{0, . . . , p$-$ 1\}$,
the hash function is defined by:
\[ h(x) = ( ( a_{k-1} x^ {k - 1} + ... + a_1 x + a_0) \,\, mod \,\, p) \,\, mod \,\,  m\]


For $p \gg	 m$, the hash function is statistically close to $k$-wise independent. The 2-wise independent hash function is:
\[ h(x) = (((a_1 x + a_0) \,\, mod \,\, p) \,\, mod \,\, m ) \]

%XXX Example for the propositional case XXX \\

Higher the $k$, stronger will be the guarantee on a range of the size of cells. 
%
To encode $k$-wise independence we will require polynomial of degree $k$ - $1$.
%
But higher the $k$, harder it will be to solve the formula with these constraints. 
%
To balance this trade-off we use 3-wise independent hash functions.
%
If we use $k$-wise independence, all cells will be small and we will get a \textbf{uniform generation}. 
%
Using $3$-wise independence we achieve \emph{a random cell is small with high probability}, known as \textbf{almost-uniform generation}.

\section{Algorithm} \label{approxasp}We use the already evolved algorithm from propositional case \cite{ChakrabortyMV13} and lift it to ASP.
%
We use 3-wise independent hashing functions to partition the set of models of an input formula into ``small" cells.

%The approximate model counting algorithm 

We assume that that the algorithm has access to \xampler\ that takes an input an ASP program $\fff^{\prime}$ possibly in conjunction with xor constraints, as well as a bound $b\geq0$.
%
The \xampler $(\fff^{\prime},b)$ returns set of models $S$ of $\fff^{\prime}$ such that $|S| = min(b, \#\fff^{\prime})$.
%
%In order to test whether the generated cells are indeed small, we choose a random cell and check if it is non-empty and has no more than pivot elements, where the pivot is a threshold that depends only on the tolerance bound \ep. 
%%
%If the chosen cell is not small, we randomly partition the set of models into twice as many cells as before by choosing a random hashing function from the family Hxor(n, m + 1, 3).
%%
%The above procedure is repeated until either a randomly chosen cell is found to be non-empty and small, or the number of cells exceeds 2
%n+1 pivot.
%%
%If all cells that were randomly chosen during the above process were either empty or not small, we report a counting failure and return $\bot$.
%%
%Otherwise, the size of the cell last chosen is scaled by the number of cells to obtain an \ep-approximate estimate of the model count.

Algorithm \appasp takes as input ASP program \fff, a tolerance \ep $(0 < \epsilon \leq 1)$ and a confidence \del $(0 < \delta \leq 1)$ as an input.
%
It computes a threshold pivot that depends on \ep to determine the chosen value of the size of a small cell.
%
Then it checks if the input program \fff has at least a pivot number of solutions.
%
It uses the \xampler\ to check if the input program has at least $b \, (b = pivots+ 1)$ solutions.
%
If the total number of solutions $S$ of \fff is less than or equals to $b$ the algorithm returns the answer set count $|S|$. 
%
Otherwise, the algorithm continues and calculates a parameter $iter (\geq 1)$ determines the number of times \countas is invoked.
%
Note that $iter$ depends only on \del.
%
Next, there are at most $iter$ number of calls are made to \countas.
%
The resulted in non-$\bot$ estimates of the ASP model count $c$ returned by \countas are appended to the list $C$. 
%
The final estimate of the model count returned by \appasp is the mean of the estimates stored in $C$, computed using FindMean(C).

\begin{algorithm}[t]
	\textbf{ApproxASP}(\fff, \ep, \del) \;
	\SetAlgoLined
	\KwResult{Approximate number of answer sets or $\bot$}
	$counter = 0$ ; $C = [\,]$ \;
%	$$\;
	$pivot =  2 \, \times \lceil 3 e^{1/2} (1 + \frac{1}{\epsilon})^2 \rceil$ \;
	$S = Xampler(\fff, pivot + 1)$ \;
	\eIf{ $|S| \leq pivot$}
	{ % Open If
		\textbf{return} $|S|$
	} % Close If
	{ % Open Else
		$iter= \lceil 27 \log_2 (3/\delta) \rceil$\;
		\While{$counter < iter$}{
			$c = $ \textbf{CountAS($\fff, pivot$)} \;
			$counter = counter + 1$ \;
			\If{$c \neq \bot$}{\textbf{AddToList($C,c$)}}
		} % Close While
	} % Close else
	\textbf{return} \textbf{FindMean($C$)} \;
	
	\caption{ApproxASP}
\end{algorithm}
%
%\begin{algorithm}[h]
%	\textbf{ApproxASP}(\fff, \ep, \del) \;
%	\SetAlgoLined
%	\KwResult{Approximate number of answer sets or $\bot$}
%	$counter \= 0$ ; $C \= emptyList$\;
%	$threshold = \lceil 3 e^{1/2} (1 + \frac{1}{\epsilon})^2 \rceil$\;
%	$iterations = \lceil 27 \log_2 (3/\delta) \rceil$\;
%	$pivot \=  2 \, \times $ \textbf{ComputeThreshold($\epsilon$)} \;
%	$S \= Xampler(\fff, pivot + 1)$ // {Solve pivot+1 AS}  \;
%	\eIf{ $|S| \leq pivot$}
%	{ % Open If
%		\textbf{return} $|S|$
%	} % Close If
%	{ % Open Else
%		$t \= $ \textbf{ComputeIterCount($\delta$)}
%
%		\While{$counter < t$}{
%			$c \= $ \textbf{ApproxASPCore($\fff, pivot$)} \;
%			$counter \= counter + 1$ \;
%			\If{$c \neq \bot$}{\textbf{AddToList($C,c$)}}
%		} % Close While
%	} % Close else
%	$finalCount \= $ \textbf{FindMean($C$)} \;
%	\textbf{return} $finalCount$
%
%	\caption{ApproxASP}
%\end{algorithm}


%\begin{algorithm}
%	\textbf{return} $\lceil 3 e^{1/2} (1 + \frac{1}{\epsilon})^2 \rceil$
%	\caption{ComputeThreshold($\epsilon$)}
%\end{algorithm}
%
%\begin{algorithm}
%	\textbf{return} $\lceil 27 \log_2 (3/\delta) \rceil$
%	\caption{ComputeIterCount($\delta$)}
%\end{algorithm}

\begin{algorithm}[t]
	/* Assume $z_1 , ..., z_n $ are the variables of \fff */ \\
	$ i, l = \lfloor \log_2 (pivot) - 1 \rfloor$ \\
%	$ i \= l - 1$ \\
	
%	$ l \= \lfloor \log_2 (pivot) - 1 \rfloor$ \\
%	$ i \= l - 1$ \\

	\While{($1 \leq |S| \leq pivot $) $\lor (i = n)$}{
		$ i = i + 1$ \\
		$h \gets H_{xor}(n, i - l, 3)$ \\
		$\alpha \gets \{0,1\}^{i-l}$ \\
		$S = Xampler(\fff \land (h(z_1, ..., z_n)=\alpha, pivot + 1)$ %\Comment*[r]{Solve $P \land XOR$}
	}

	\eIf{ $|S|$ $> pivot$ or $|S|$ = 0}
	{\textbf{return} $\bot$}
	{\textbf{return} $|S|$ $\cdot 2^{i-l}$}



	\caption{CountAS($\fff, pivot$)}
\end{algorithm}

Algorithm \countas takes as input an ASP program \fff and a
pivot.
%
It returns an \ep-approximate estimate of the model count of the program \fff.
%
The algorithm uses random hash functions from $H_{xor}(n, i-l, 3)$ to partition the model space of the program \fff.
%
This is done by choosing a random hash function $h$ (on line 5) and choosing randomly with probability half ($\alpha$, line 6) bits to set.
%
Then it conjunct the chosen xor with the program \fff and use \xampler\ to check whether it has at most $pivot +1$ models.
% 
This process repeats (lines 5-7) and the loop terminates if either a randomly chosen cell is found to be small ($|S| \leq pivot$) and non-empty, or if the number of cells generated $> 2^n+1/pivot$.

We scale the size of $S$ by the number of cells generated by the corresponding hashing function to compute an estimate of the model count.
If all randomly chosen cells were either empty or not small, we return $\bot$ and report a counting error.

\section{Experiments} \label{sec:experiments}

To test the approximate counting in ASP, we extend \xorro{} with the algorithms described in the section above.
For now, we focus only on the quality of the counting, leaving the scalability and performance for further work.
%
To test the quality of the counting,
we generated ten random instances from different ASP problem classes such as Graph Coloring, Vertex Cover, Minimal Vertex Cover, and One More.
Also, to track the counting, we cared that these instances were ``easy to solve'' for \clasp{},
meaning that \clasp{} must enumerate all answer sets within 600 seconds timeout (without printing).

To get the feeling for our initial counting experiments, we prefer to keep the number of variables low, so we fixed the confidence and the tolerance values to 0.8, and 0.2 respectively, as shown in [Meel].
In other words, we kept the same pivot value seeking for a "small" cell and the number of iterations fixed, as shown in lines 3 and 8 from Algorithm 1, respectively.
As part of the setup of the experiments, we ran \xorro{} calculating the approximate counting not only with the median value but also with the mean. 

The experiments were run sequentially under Linux on a 16 GB memory with a 2.60 GHz Dual-Core Intel Core i7 processor laptop.
Each benchmark instance (in smodels output format, generated offline with \gringo{}) was five times and restricted to 600 seconds.
As shown in Algorithm 1, a run is finished with one of three possible situations, either \xorro{} returns the approximate answer sets count or unsatisfiability, or without the count due to time exhaustion.

Our experiments' results are summarized in Tables 1-TBD listing for each problem class instances, the number of answer sets in the first two columns.
The remainder of the table is divided into the best and worst runs from the five. 
%
For both cases, we list the median and the mean counts, each with a quality factor (Q) estimating the closeness to the total number of answer sets.
The last row of each table displays the average Q for each count.

\begin{table*}[t]
  \centering
  \scalebox{0.9}{
\begin{tabular}{c|r|@{\qquad}rrrr|@{\qquad}rrrr}
  \toprule
   &  & \multicolumn{4}{c}{Best} & \multicolumn{4}{c}{Worst} \\
  %\cmidrule{2-7}
  init & \#Answer Sets &  Median & Q & Mean & Q & Median & Q & Mean & Q \\
  \midrule
  1  & 262,080    & 199,680   & 0.76 & 254,586    & 0.97 & 150,528   & 0.57 & 252,672    & 0.96\\
  2  & 90,000     & 99,584    & 1.11 & 105,204    & 1.17 & 41,472    & 0.46 & 64,771     & 0.72\\
  3  & 62,400     & 53,504    & 0.86 & 64,127     & 1.03 & 52,224    & 0.84 & 68,359     & 1.1\\
  4  & 37,680     & 33,024    & 0.88 & 37,683     & 1    & 20,480    & 0.54 & 34,798     & 0.92\\
  5  & 70,560     & 54,144    & 0.77 & 70,332     & 1    & 43,520    & 0.62 & 58,700     & 0.83\\
  6  & 4,800      & 3,312     & 0.69 & 4,882      & 1.02 & 3,192     & 0.67 & 4,781      & 1\\
  7  & 20,880     & 19,456    & 0.93 & 25,817     & 1.24 & 19,264    & 0.92 & 25,796     & 1.24\\
  8  & 9,959,040  & 5,586,944 & 0.56 & 9,855,381  & 0.99 & 4,456,448 & 0.45 & 10,147,541 & 1.02\\
  9  & 13,996,920 & 6,897,664 & 0.49 & 15,501,920 & 1.11 & 5,931,008 & 0.42 & 11,033,737 & 0.79\\
  10 & 5,569,560  & 3,407,872 & 0.61 & 6,585,568  & 1.18 & 3,391,488 & 0.61 & 6,222,679  & 1.12\\
  \bottomrule
     & AVG &  & 0.77 &  & 1.07 &  & 0.61 &  & 0.97\\
  \bottomrule

\end{tabular}}	
  \caption{Approximate answer set count over random instances of the Graph Coloring problem.}\label{table:graph_color}
\end{table*}


\begin{table*}[t]
  \centering
  \scalebox{0.9}{
\begin{tabular}{c|r|@{\qquad}rrrr|@{\qquad}rrrr}
  \toprule
   &  & \multicolumn{4}{c}{Best} & \multicolumn{4}{c}{Worst} \\
  %\cmidrule{2-7}
  init & \#Answer Sets &  Median & Q & Mean & Q & Median & Q & Mean & Q \\
  \midrule
  1  & 46,737      & 46,592      & 1    & 45,171      & 0.97 & 36,224      & 0.78 & 50,838      & 1.09\\
  2  & 58,538      & 55,296      & 0.94 & 60,324      & 1.03 & 40,448      & 0.69 & 49,216      & 0.84\\
  3  & 6,405,610   & 6,193,152   & 0.97 & 6,449,549   & 1.01 & 5,701,632   & 0.89 & 7,413,423   & 1.16\\
  4  & 5,330,500   & 5,210,112   & 0.98 & 5,174,303   & 0.97 & 3,604,480   & 0.68 & 4,619,493   & 0.87\\
  5  & 7,460,775   & 7,536,640   & 1.01 & 7,446,894   & 1    & 6,717,440   & 0.9  & 6,787,805   & 0.91\\
  6  & 5,733,125   & 5,734,400   & 1    & 5,811,551   & 1.01 & 4,620,288   & 0.81 & 5,966,710   & 1.04\\
  7  & 187,928     & 183,296     & 0.98 & 188,198     & 1    & 155,648     & 0.83 & 195,252     & 1.04\\
  8  & 919,808     & 909,312     & 0.99 & 916,914     & 1    & 962,560     & 1.05 & 963,723     & 1.05\\
  9  & 493,431,189 & 492,830,720 & 1    & 494,853,532 & 1    & 353,370,112 & 0.72 & 440,323,668 & 0.89\\
  10 & 5,785,344   & 5,799,936   & 1    & 5,868,650   & 1.01 & 6,307,840   & 1.09 & 6,407,616   & 1.11\\
  \bottomrule
     & Average     &             & 0.99 &             & 1    &             & 0.84 &             & 1\\
  \bottomrule

\end{tabular}}	
  \caption{Approximate answer set count over random instances of the Vertex Cover problem.}\label{table:vertex_cover}
\end{table*}


% comment results


The large deviations between the best and the worst cases correspond to one of two possible scenarios.
%
If the worst-case count is under approximating,
it means the partition was not well distributed, and some clusters had too few or too many answer sets. 
%
On the opposite case, where there is an over-approximation count,
our set of \XOR{}s contains linear combinations or linearly dependent equations,
meaning that the partitioning is not performed concerning the number of \XOR{}s.
% 
For instance, the conjunction of the \XOR{} constraints $a \xor \top \land b \xor \top \land a \xor b \xor \top$ can be equivalently reduced to $a \xor \top \land b \xor \top$. 
%
Back to our example in Section~\ref{sec:parity}, instead of counting $|S|$ $\cdot 2^5$ being $S=2$,
one linear combination causes the double of answer sets from the resulting cluster, so for this case, $S=4$, and the approximate count is 1024 instead of 64. 



As we mentioned above, the performance was not examined for this paper,
meaning that it is worth considering for further experiments by testing all the different approaches from \xorro{}.
For the experiments above, we ran \xorro{} with the lazy counting approach,
which got the highest overall performance score from all the six implementations.
However, the random parity constraints generated during each counting iteration were quite small,
meaning that other approaches would benefit more for these \XOR{}s densities.

\section{Conclusion and Future work} \label{sec:conclusion}
This paper discusses an extension of the ASP solver \xorro{}
towards approximate answer set counting. This is established by lifting ideas from existing techniques for SAT to the formalism of ASP.
While our preliminary results are promising and show that
indeed approximate counting works for ASP, there is still potential for future works.
On the one hand, we highly recommend studying and showing
proper guarantees that prove our results are guaranteed to be accurate with high probability and do not deviate far from the actual result.
Further, we highly encourage further tuning and improvements of our preliminary implementation.
We hope that this work fosters applications and further research on quantitative reasoning for ASP.

%% The file kr.bst is a bibliography style file for BibTeX 0.99c
\bibliographystyle{kr}
\bibliography{procs,NMR2020}

\end{document}

